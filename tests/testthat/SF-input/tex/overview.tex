\begin{overview}

A new approach can resolve Australia's fifty-year debate about how to fund our schools.
Commonwealth and state education ministers are meeting in two weeks' time to discuss a new funding model for 2018 and beyond.
They can agree to radical but achievable change.

This report proposes a compact for the needs-based funding system all main parties say they want.
It gives money to the schools that need it most.
And it kick-starts transformation of teaching and learning in all schools, investing in new roles for expert teachers to lift student performance. Critically, the compact does not require more Commonwealth funding, although some state governments may need to step up.

School funding needs a circuit breaker.
The needs-based model recommended by the 2011 Gonski Review was widely supported but not delivered in practice.
The trajectories in the 2013 Education Act are too slow: many under-funded schools will not be properly funded for decades, while other schools will still be over-funded at the \emph{end of the century}.

The legislated approach is also too costly.
It locks Australia into long-term funding growth rates that are too high given low wages growth.
In addition, Labor's promise that ``no school will lose a dollar'' entrenched decades of special deals done by both sides of politics.
To fund all schools according to need under this model, governments would have to spend around \$3.5~billion more every year.

The Coalition's 2016 Budget planned to cut the costs of the 2013 Act.
But the enacting the budget probably requires legislative change which have not been -- and may not be -- passed by Parliament.
The full details of the 2016 Budget plan are still unclear, but it appears
to create new problems.

The new compact shows how to deliver needs-based funding without the spending increases required by the 2013 Act.
With ongoing low inflation, school indexation rates are billions of dollars more generous than they need to be.
The compact seizes this historic opportunity.
It opens up large savings by reducing indexation rates to line up with wages growth.
It would then reallocate these funds to achieve needs-based funding by 2023.

Under the compact, very under-funded schools would be much better off compared to the 2013 model.
Chronically disadvantaged schools benefit the most.
Almost half of schools at or just below their targets would have slower funding growth, but they would maintain their purchasing power from today.
A small number of over-funded schools would lose money.
Compared to the 2016 Budget, most schools would be better off.
To avoid a similar mess in future we recommend greater transparency in funding.
These changes are vital to creating a school system that gives all children a fair chance in life.

Funding is not enough.
It must be accompanied by broader reforms to improve teaching and learning.
Investing in effective teaching does the most to lift student outcomes.
The compact redirects a big part of the savings relative to the 2013 Act to create two new teaching roles. \emph{Master Teachers} and \emph{Instructional Leaders} will work in and across schools to drive improvements in teaching effectiveness in their subject areas.

It is time to end the toxic funding debate so that we can focus on the debate that counts in this century: how do we equip our teachers to improve the learning of all students?
\end{overview}