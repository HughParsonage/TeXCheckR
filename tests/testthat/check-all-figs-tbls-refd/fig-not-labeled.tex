\documentclass{grattan}
% Comments are deployed by the % sign; everything after % is ignored by the compiler.
% Please do not put comments before \documentclass as these are reserved for TeX directives.

\addbibresource{bib/Grattan-Master-Bibliography.bib}

\author{}
\title{}

\GrattanReportNumber{2017-00}

\acknowledgements{%
This 
%
}




\begin{document}


\begin{overview}
This is the overview. 
The word \textbf{Overview} appears in the contentspage at \verb=\contentspage= as an unnumbered entry.
Set over two columns, the first and last baseline of the text in each column have the same vertical positions.

(But only if the overview is large enough!
If the text is too brief, the algorithm might prefer some other arrangement.
This is unlikely.) 

The first line of the overview starts at a fixed vertical position but the height of the text is determined by the constraint of matching the baselines.
In particular if the overview is too large, it will overflow the page.

Footnotes are not permitted in the overview.
\end{overview}

\begin{recommendations}
\recommendation{Recommendation}
Some text under a recommendation.
\end{recommendations}

\contentspage

\chapter{Chapter title}\label{chap:example}
This demonstrates the markup commands for sectioning. 
If you want to cross-reference a chapter/section/\etc\ later, you need to add a label, as above and below.
Although \LaTeX{} doesn't require it, every label must have a specific prefix followed by a colon, as throughout this document.
Furthermore, every chapter must be labelled, even if it is never referenced.

To refer to a chapter, use \Chapref{chap:example}, % not \Vref{chap:example}, 
otherwise the hyperlink points to the baseline of the first text of the chapter, rather than the chapter title.
To refer to a chapter range, \Chaprefrange{chap:example}{chap:citations}.

\section{Section title}\label{sec:example}
\subsection{Subsection title}\label{subsec:example}
\subsubsection{Sub-subsection title}\label{subsubsec:example}
Some text in one paragraph. 
Some text in one paragraph. 
Some text in one paragraph. 

Some text in the next paragraph. 
Some text in the next paragraph. 
Some text in the next paragraph. 

Since single newlines are ignored, put different sentences on different lines.

\begin{enumerate}
	\item An enumerated list
	\item Another item
	\begin{enumerate}
		\item A sub-item of the second item
		\item Another sub-item

		New paragraphs are the same as in the body text
	\end{enumerate}
\end{enumerate}

Another list:
\begin{itemize}
	\item A bulleted/non-enumerated list
	\item Another item
	\begin{itemize}
		\item A sub-item.
	\end{itemize}
\end{itemize}

\chapter{Figures and tables}\label{chap:figs-and-tbls}
An initial cross-reference to \Vref{fig:first-example-figure}.

\begin{figure}
\caption{Figure caption\label{fig:first-example-figure}}
\units{Figure units}
\includegraphics{logos/GrattanSVGLogo.pdf}
\noteswithsource{Notes}{Source}
\end{figure}

A cross-reference to \Vref{box:example}.

\begin{verysmallbox}{A very small box}{box:very-small-box-example}
Contents of very smallbox.

A paragraph with a footnote.

Some text in a very small box. 
Some text in a very small box. 
Some text in a very small box. 
\end{verysmallbox}

\begin{smallbox}{A small box}{box:example}
A smallbox. 
Some text in a small box. 
Some text in a small box. 
Some text in a small box. 
Some text in a small box.%
\footnote{Footnote entry.}


Some text.%
\footnote{All footnotes must end with a full stop.} 
Some other text.%
\footnote{(Or a closing parenthesis, which itself must be preceded by a full stop.)}
To refer to a footnote, use footnote~\ref{fn:example} \vpageref{fn:example}; % rather than \Vref{fn:example} which will link to a different counter; 
the label command \emph{must} be inside the footnote.%
\footnote{\label{fn:example}For example, refer here.}
\end{smallbox}

A cross-reference to \Vref{fig:2nd-example-figure}, with a different cross-reference that will never print the page number: \Cref{fig:first-example-figure}.

If a sentence ends with a capital letter, you must use backslash-@ immediately before the full stop.
Many governments have tried to change the GST\@. 
But few have succeeded.

\begin{figure}
\caption{A second figure caption\label{fig:2nd-example-figure}}
\units{Figure units}
\includegraphics{logos/GrattanSVGLogo.pdf}
\noteswithsource{Notes}{Source}
\end{figure}

A table:

\begin{table}
\caption{Table caption}\label{tbl:one-table}
\begin{tabularx}{\linewidth}{XXR}
%
\toprule
\textbf{Column title}                           & \textbf{Column type}                              & \textbf{Ragged left title} \\
\midrule
Move from one cell to the next cell in the same row by using an ampersand & Move to the next line by using a double backslash & Use toprule, midrule, and bottomrule. \\
Move from one cell to the next cell in the same row by using an ampersand & Move to the next line by using a double backslash & Use toprule, midrule, and bottomrule. \\
Move from one cell to the next cell in the same row by using an ampersand & Move to the next line by using a double backslash & Use toprule, midrule, and bottomrule. \\
Move from one cell to the next cell in the same row by using an ampersand & Move to the next line by using a double backslash & Use toprule, midrule, and bottomrule. \\[15.5pt]
Use the optional argument & to the double backslash & to specify the precise distance between rows. \\
\cmidrule(lr){2-3}
Never use vertical rules & But sometimes horizontal rules are appropriate. & Use cmidrule(lr) \\
\bottomrule
\end{tabularx}
\noteswithsource{Notes}{Source}



\end{table}
Another table, cross-referenced in the same way \Vref{tbl:one-table}.

A big box, occupying two pages floats with respect to the body text.




\end{document}
