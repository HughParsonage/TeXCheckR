\GrattanReportNumber{2017-00}

Goss, P., Sonnemann, J., and Griffiths, K. (2017). \emph{\mytitle}. Grattan Institute.

In this report we look at one of these 'other' elements of effective teaching -- creating a classroom environment that gives all students the best opportunity to learn. A good learning environment raises student expectations, encourages them to participate, and ensures that no student can fly under the radar.  
footnote: Marzano, et al. (2003); 

Evertson and Weinstein (2006); 

Porter (2007); McDonald (2013); Jones and Jones (2004)

{Hattie (2009)}
A major 2009 study shows that interventions aimed at improving the classroom climate, such as lifting student and teacher expectations, teacher clarity, improving teacher-student relations, classroom behaviour, classroom management were especially effective (see \Vref{fig:classroom-environment-factors}). 
Empirical studies consistently show that well-managed classrooms enhance student behaviour, engagement and achievement. footnote: Marzano, et al. (2003); Oliver, et al. (2011) ? effect sizes are: 0.71 for student behaviour, 0.62 for student engagement and 0.52 for student achievement.
It is a necessary skill set for effective teaching and learning. %Evidence shows that more effective teachers are better at engaging and managing students in the classroom. For example, the US Measures of Effective Teaching (MET) project evaluated 3000 teachers and found that more effective teachers are better at developing an effective learning environment Kane and Cantrell (2010)
It also matters for teachers. The classroom environment can also have a big impact on the teacher?s job satisfaction. Indeed, teacher-student relationships are the most important factor influencing teachers? job satisfaction and sense of efficacy.  footnote: Freeman, et al. (2014) ? Table 7.3, p.144
When student behaviour is continuously challenging, teachers struggle. How to hand difficult student behaviour is consistently rated as a leading cause of teacher stress and burnout  %Footnote: Lewis, et al. (2005); Stoughton (2007)

and burnout can lead to a teacher giving up and leaving the profession. %Footnote: Goddard and Goddard (2006)
The teacher?s ambition should not necessarily be a quiet classroom, but a genuinely productive class. footnote: Watkins and Noble (2013)

Developing students? skills of self-regulation, resilience and tolerance for others are key goals of 21st century schooling. A focus on the ?whole student? ? through social, emotional and intellectual development ? is a key priority at school. Footnote: Cain & Carnellor, 2008; McCombs, 2004; Noddings, 1995; Palmer, 2003), cited in Waters (2011)

What the teacher does, particularly in the early years of schooling, plays a big role in developing students? broader skills for at school and work. footnote: Watkins (2005); Watkins and Noble (2013)

Major studies consistently identify a number of evidence-based approaches that teachers should adopt for an effective learning environment in the classroom. footnote: Simonsen, et al. (2008); Hattie (2009); Marzano, et al. (2003)
A mix of preventive and responsive approaches is needed.  %Footnote: Other groupings exist, for example, Marzano?s seven elements (Marzano, et al. (2003)) and the National Council on Teacher Quality?s ?Big Five? (Greenberg, et al. (2014)).
A major 2009 study in Western Australia which tracked 1,300 students found that about 40 per cent of students displayed unproductive behaviours regularly (\Vref{fig:40-pc-regularly-unproductive}). footnote: The Pipeline Project ? tracked 1,300 students over four years in 31 schools in Western Australia, with a particular focus on low socio-economic schools, by Angus, et al. (2009)

In a 2014 study in South Australia, teachers similarly reported widespread problems with a lack of engagement. footnote:The Behaviour at School Study is a survey of 1,380 teachers in schools across South Australia, by Sullivan, et al. (2014)
\noteswithsource{Percentage of students productive vs. unproductive is averaged across 4 years (2005-2008). 10 types of unproductive behaviours and their frequency were reported for one year (2008) and we classified these into ?disengaged?, ?disruptive? or ?aggressive?.}%
{Angus et al. (2009)}
\noteswithsource{Percentage of students productive vs. unproductive is averaged across 4 years (2005-2008). 10 types of unproductive behaviours and their frequency were reported for one year (2008) and we classified these into ?disengaged?, ?disruptive? or ?aggressive?.}%
{Sullivan et al. (2014)}
A large 2014 South Australian study shows that remote and low socioeconomic (SES) schools have higher rates of disengagement and low-level disruption. More than 60 per cent of teachers in low SES schools report regular daily disruption in class, whereas only 10 per cent in high SES schools report the same problem (see \Vref{fig:unproductive-behaviours-low-SES} and Appendix A %!!! cross ref


).footnote:Sullivan, et al. (2014)

It is important to note, however, that this does not necessarily mean poor behaviour is inherent to students from low SES backgrounds; indeed, poor student behaviour is not always a feature of disadvantaged schools. footnote: Some low SES schools have much better than expected levels of engagement with schoolwork ? Angus, et al. (2009)

Higher rates of misbehaviour may reflect problems at home, or relate to the higher proportion of early career teachers in these schools. footnote:Freeman, et al. (2014)
\noteswithsource{Percentage of students productive vs. unproductive is averaged across 4 years (2005-2008). 10 types of unproductive behaviours and their frequency were reported for one year (2008) and we classified these into ?disengaged?, ?disruptive? or ?aggressive?.}%
{Source: Sullivan et al. (2014)}

A handful of studies point to some students being bored and unable to engage in Australian classrooms. Two large international studies (one quantitative, one qualitative) evaluate students? reasons for misbehaving and non-participation. The top reasons in the quantitative study were boredom, attention-seeking, and work-related difficulties (students didn?t believe they could do it, so they didn?t try). The qualitative study also identified boredom, as well as teacher-student misunderstanding and students? negative attitudes towards school. footnote:Reported in Montuoro and Lewis (2015). These findings are in keeping with an older 2001 Queensland School Reform Longitudinal Study which found that maths and science pedagogies and assessment tasks were not intellectually demanding enough to engage many students- see Lingard, et al. (2001)

In the Western Australian study that tracked the same students over four years, unproductive students were on average one to two years behind their peers in literacy and numeracy. On average, students who behaved productively performed significantly higher than those who behaved unproductively (see Figure 3). footnote:Angus, et al. (2009) This holds for reading and numeracy findings in this study 

A surprising finding is that compliant but quietly disengaged students do just as poorly, on average, as disruptive students, who are likely to be drawing attention to themselves (see \Vref{fig:lower-scores-on-average}).  footnote: Angus, et al. (2009)

{Angus et al. (2009)}

Quiet disengagement can take various forms. footnote:Galton (1999). The original study, led by Maurice Galton in 1976, comprised thousands of observations of 489 primary school children and 58 teachers in the UK over three years. Galton?s follow-up study in 1996 found many of the same behaviour types.

?Easy riders? give the appearance of working but do so more slowly than other students, finding ways of extending routine tasks without attracting the teacher?s attention. This is particularly problematic because it lowers teachers? expectations of the student.footnote:Angus, et al. (2009); Galton (1999)

Each student?s learning can be influenced ? positively and negatively ? by their peers.  Peer misbehaviour, for example, may disrupt the class, while positive peer behaviours such as studying for a test can improve a student?s own behaviour and learning. footnote:McVicar, et al. (2013), Carrell, et al. (2016)

Students are not passive spectators of events in their classroom, so the teacher?s response to one misbehaving student can create a ?ripple effect?. footnote:McVicar, et al. (2013)

In the 2014 South Australian study, teachers reported being highly stressed by student behaviour. Nearly one in three teachers reported being ?extremely stressed? or ?very stressed? by the challenges of engaging and re-engaging students in class.  footnote: Sullivan, et al. (2014)

\source{Sullivan et al, 2014.}

Student behaviour is a prominent concern in teacher surveys, particularly among trainee and new teachers. footnote:Boyle, et al. (1995); Giallo and Little (2003); Stoughton (2007); Richards (2012); Feltoe (2013); Buchanan, et al. (2013)

These issues are consistently rated as a leading cause of teacher stress and burnout.footnote:Lewis, et al. (2005); Stoughton (2007)

In addition, teachers in low SES and remote schools are particularly stressed by student behaviour. Disruptive and disengaged behaviours are reported more frequently in these schools, for a variety of reasons, as discussed in Section X (see Appendix A). footnote:  Sullivan, et al. (2014). In addition, casual and part-time teachers also call for extra support in engaging and managing students Buchanan, et al. (2013); TEMAG (2014). Mayer, et al. (2014)

Other survey data shows that over a quarter of experienced teachers also perceive a need for further development in this space.  footnote:McKenzie, et al. (2014)

\noteswithsource{Categories reflect the AITSL Graduate Teacher Standards.}{McKenzie et al. (2014)}%

\source{Sullivan et al. (2014)}%

Several studies show that teacher responses to disengaged and disruptive behaviours in Australian classrooms could be better. footnote:Sullivan, et al. (2014); Lewis, et al. (2008); Montuoro and Lewis (2015)

Teacher-student misunderstandings may arise in part because what teachers believe and what they do are not always in sync. footnote:  Fang (1996)

While teachers believe in giving warnings and explanations, an Australian survey of students who had been sent out of the classroom showed that only 30 per cent believed they had been given a warning before they were excluded. Only 40 per cent reported that the teacher gave them a reason for their removal. footnote:Lewis, et al. (2012). 

\source{Study 1 --- Clunies--Ross et al. (2008) survey of 97 primary teachers from Melbourne. 

Study 2 --- Wheldall and Beaman (1994) observation of 36 primary teachers from seven Sydney schools. 

Study 3 --- Wheldall and Beaman (1994) observation of 79 secondary teachers from four Sydney schools, reported in Beaman and Wheldall (2000).}

Aggressive behaviour by teachers ? including whole-class punishments, yelling, humiliating students or being sarcastic towards students ? distracts students and diminishes both academic and social-emotional learning. footnote:Riley and Brew (2010), Lewis (2001); Lewis, et al. (2008) 
Despite this, teacher aggression towards students still occurs in many Australian schools, albeit infrequently. A 2005 study of teachers? classroom management strategies, drawing on the views of more than 4,000 students and nearly 500 teachers, reported that some teachers were aggressive ?some of the time?. 
The study found that such aggressive behaviours distracted students and made them feel worse towards their teacher. footnote: Lewis, et al. (2008) 

In a more recent survey of almost 500 Victorian teachers, most reported using aggressive strategies ?hardly ever? or ?never?. But a third conceded that they had yelled angrily at students, at least sometimes. And almost half reported making sarcastic comments to students who misbehave, sometimes or most of the time. footnote: Romi, et al. (2011)

A common theme is that teacher aggression emerges when the teacher is stressed and tired: \emph{?I don?t mean to ? they just happen?}. footnote:Riley and Brew (2010)

Stressed teachers need support but may be the least likely to ask for help. footnote: Lewis (2001)

Poor-performing teachers tend to express the belief that factors totally outside their control ?determine? student outcomes. footnote:Lingard, et al. (2001)

One study estimates that almost 10 per cent of Australian teachers work in schools where intimidation or verbal abuse of staff by students occurs on a weekly basis. This rate is much higher than other countries surveyed (average of 3.4 per cent).footnote:This data is reported by the school principal. Freeman, et al. (2014)
Several big studies highlight specific evidence-based techniques for creating an effective learning environment. %Footnote:   Key meta-studies used in this chapter include: Simonsen, et al. (2008); Hattie (2009); Marzano, et al. (2003), (Greenberg, et al. (2014)). These include evidence from Australia and US settings (CHECK REF !!! ).
As always, prevention is better than cure. The ability of teachers to quickly and accurately identify behaviour that might become problematic (sometimes called ?with-it-ness?) is critical. footnote: Marzano, et al. (2003)

In this chapter we synthesize the common approaches highlighted in the research. They are:footnote:Other groupings exist, for example, Marzano?s seven elements (Marzano, et al. (2003)) and the National Council on Teacher Quality?s ?Big Five? (Greenberg, et al. (2014)

Effective teachers instil in every student an expectation of success. They recognise that student motivation, engagement and self-belief can drive student achievement ? and vice versa. footnote:OECD (2013) 

When students achieve success, their self-esteem lifts and they become more engaged, which leads to ever better performance. Competence breeds self-esteem and confidence, which in turn breeds greater competence. footnote:Porter (2007); Brophy (2013) 

One strategy teachers can use to lift student expectations is to recognise early achievements and strengths. Student expectations of their own performance draw on their prior achievements. footnote:Hattie (2009)

The expectations of a child are \emph{?powerful enhancers of ? or inhibitors to ? school education?}. footnote:Hattie (2009) p.31  
Teacher bias in expectations can also be self-fulfilling. For example, high expectations for a student could translate into more school and teacher resources being devoted to that student, or more effort being put in by that student. The reverse can also be true: low expectations can translate into fewer resources and less effort. footnote:Recent work by the US Institute of Labour Economics shows that teacher expectations have a causal impact on achievement https://www.brookings.edu/blog/brown-center-chalkboard/2016/09/16/do-teacher-expectations-matter/
(see \Vref{fig:high-priority-professional-learning}). footnote:Hattie (2009); Marzano, et al. (2003) 

The relationship a teacher has with their students can be the difference between students accepting or resisting classroom rules and procedures that enable learning. footnote: Marzano, et al. (2003)

A good relationship helps teachers to tailor effective intervention strategies when behaviour and learning problems arise. footnote: Epstein, et al. (2008)

And such strategies are more likely to work when the relationship is good.  footnote:Marzano, et al. (2003); Montuoro and Lewis (2015)

It is not about whether students ?like? their teachers. The studies show that the best teacher-student relationships require the teacher providing strong guidance and showing clear purpose as well as concern for the needs of others and a desire to work as a team.  footnote:Marzano, et al. (2003)

Mutual respect is important; teachers should recognise students? rights to learn, to feel emotionally and physically safe, and to be treated fairly. footnote:Rogers (2015); Lewis (2011) 

Empathy is vital, but strong relationships also require teachers being able to maintain ?a healthy emotional objectivity? with students. footnote:Hattie (2009), Marzano, et al. (2003), Personalising student misbehaviour can sometimes drive inappropriate teacher responses, for example teacher aggression - Riley and Brew (2010)

Teachers must be clear and consistent about what students are expected to do, as well as teaching them how to do it. footnote:Marzano, et al. (2003); Simonsen, et al. (2008)

(see Figure 8). footnote:Hattie (2009) ; footnote: Direct instruction (where teachers model then lead students through content and finally test student knowledge of the content) is considered by some an effective strategy for increasing on-task behaviour, see Simonsen, et al. (2008)

(see \Vref{box:explicit-modelling-learning-behaviours}). footnote:Brophy (2006)

Explicitly stating the learning goals, defining and explaining classroom procedures, directing activities, and minimising distractions are among the things teachers can do to help create the best classroom environment for learning. footnote:Marzano, et al. (2003); Simonsen, et al. (2008); Greenberg, et al. (2014)

Increasing the rate at which students have the opportunity to respond in class has a positive impact on achievement. footnote:Simonsen, et al. (2008) 

And opportunities to collaborate with peers and do group work improve a student?s achievement, interpersonal relationships and attitudes to learning. footnote:Marzano, et al. (2003)

Class-wide peer tutoring programs (where one student coaches another) have been shown to enhance academic engagement and reading achievement. footnote:Simonsen, et al. (2008). 

footnote:See Simonsen et al. (2008) on the value of response tools.

Responses should include a combination of approval and disapproval. One large study found that positive reinforcement was on average more effective than punishment, but that a combined of the two was most effective of all. footnote:Stage and Quiroz (1997) reported in Marzano, et al. (2003)

Positive reinforcements include praise, encouragement and rewards. Praise should be specific and sparing. footnote:Simonsen, et al. (2008)

Positive attention and genuine encouragements such as ?good job? are also valuable. A popular encouragement, known as ?catching students being good?, uses a note, cue or quiet word to subtly recognise and reinforce appropriate learning behaviours. footnote:Marzano, et al. (2005)

Giving rewards, such as tokens for privileges, can be effective. They are most effective when both given for positive behaviour and removed for negative behaviour. footnote:Marzano, et al. (2003)

But giving rewards can undermine student responsibility, and so should only be done in concert with other approaches. footnote:Bear (2015)

 Students will not maintain good behaviour in the absence of consequences, and teachers need options when things get out of control.footnote:Lewis, et al. (2013)   
 
But it is not appropriate to always jump straight to punishment without some warning, a ?correction?, which gives the student an opportunity to change their behaviour. Punishments should always have a clear learning purpose. The teacher must explain to the student why they are being punished and, in particular, highlight how the misbehaviour affects the student?s own learning and/or the learning of others. footnote:Lewis (2011); Rogers (2015) 

Corrections can reduce the prospect of behaviour getting worse and requiring punishment. Scanning the class for students not engaged and then acting quickly ? moving closer to a student, perhaps, making eye contact, or pausing ? can make a student aware of their behaviour and allow them to adjust it themselves. footnote: Marzano, et al. (2003); Rogers (2015)

Verbal corrections, such as a warning, should be brief, calm and clear about what is required.  footnote:Simonsen, et al. (2008)

Teachers often confront the balancing act of how much attention to give a misbehaving student. Teacher attention is an incentive for many students, so ?tactical ignoring? of minor issues in combination with praise for appropriate behaviour can help to promote positive learning behaviours. footnote:Simonsen, et al. (2008)

But things go wrong when particular students are routinely separated from class. The disciplinary removal of students from class can have a negative impact on student outcomes and the school learning climate. footnote:Skiba & Sprague (2008); see also: Skiba and Rausch (2006), p. 1071; Osher et al. (2010)  

The offending students will have their education interrupted, and relationships between the students and teachers at a school overall may suffer.  footnote:Kohistani et al. (2015); Skiba & Sprague (2008)

Similarly, the practice of streaming disruptive students into special classes or schools prevents students with behavioural issues from learning from the actions of their more prosocial peers. footnote:Søren (2010)

The quality of the instruction may be compromised, and the possibility for reintegration is limited. footnote:Graham (2016)

There is also a considerable risk that exclusionary practices may disproportionately punish those students who have the highest educational and social needs. footnote:Noguera 2010, p. 342; see also Skiba & Sprague (2008); Kohistani et al. (2015); Osher et al. (2010)

And these practices have long-term consequences. Suspended students are 35 per cent less likely to finish high school and are at increased risk of contact with the justice system.  footnote:Granite, 2012; Fabelo, 2011 CHECK REF!!!

Being aware of the evidence about strategies to improve the learning environment is critical. But teachers need to know not only \emph{what} to do ? but \emph{when} and \emph{how} to do it. footnote:Various models and guides exist that can help teachers to better employ evidence-based strategies in their classrooms. Guides include Marzano, et al. (2003), Epstein, et al. (2008) and Greenberg, et al. (2014). Models include  Evertson (1995), Alberto and Troutman (2012), Canter (2011), Rogers (2015), and PBIS (http://www.pbis.org/) ? see O'Neill and Stephenson (2014) and Evertson and Weinstein (2013) for model comparison.

\source{Adapted from Epstein et al. (2008)}%

For all aspects of effective teaching, high-quality professional learning goes beyond workshops and short courses and focusses on in-school activities, such as classroom observation, peer feedback and collaboration. footnote:OECD (2014); Schleicher (2016) [insert meta-analyses here] FIND REF!!!

A study of more than 2,300 teachers in 241 schools in the US found that professional learning about classroom behaviour was most effective when it was classroom-based, school-wide, and sustained over an extended period. Coaching from experts was also found to greatly enhance teaching and learning outcomes.  footnote:Hough (2011)  Need more references here FIND REF!!!

Teachers are more likely to succeed in cultivating an effective learning environment when they are supported by a common approach across the school. footnote: Epstein, et al. (2008)

Many Australian schools already have a school-wide student behaviour plan. footnote:For example, the School-Wide Positive Behaviour Support model, also known as Positive Behaviour for Learning (PBL), is promoted in most states. There are other less prescriptive approaches too, such as Ecological and Social-Emotional Learning approaches. Few of these models have been formally evaluated (Stephenson and O?Neill 2014).This report does not examine which school-wide behaviour models work best, but focuses on the processes and structures that can increase adoption of the chosen model. 

Many schools are embracing the philosophy of positive psychology in education ? the idea that when we feel good we function well and are better able to learn. It aims to build student resilience, student wellbeing and achievement. Schools are increasingly using whole-school approaches, called ?positive education?. Many use Seligman?s PERMA model. footnote: The PERMA model covers positive emotions, engagement, relationships, meaning and accomplishment. Waters (2011), based on Seligman (2011). 

There is some evidence that positive psychology interventions in schools can improve academic outcomes as well as increase wellbeing and engagement. footnote:Durlak, et al. (2011); Waters (2011).

For example, programs can aim to build character strengths such as self-discipline. There is some empirical basis for this approach: for example a longitudinal study of eighth-graders showed that self-discipline out-predicted IQ on academic performance. footnote:Duck and Seligman (2005), quoted in Waters (2011) 

Given its potential and the early signs of impact, positive education is an area to watch. However, even advocates recognise that more consistent and reliable measurement of impact is needed for positive psychology is to be taken more seriously in education policy. footnote:White (2016) ? see LINK

footnote / box source: Durlak et al. (2011); Waters (2011); Waters et al. (2015); White (2016)

New teachers need comprehensive and induction programs to help them develop these practical skills and strategies. Too many Australian schools fail to meet this obligation. Only half of Australian teachers report having participated in an induction program as a new teacher. footnote: TALIS 2013 ? Table 4.1. This is contrary to reports by Australian principals who report that most schools have a formal induction program for new teachers.

Further, only about a third of early-career teachers are reported to have a mentor. %Footnote: TALIS 2013 ? Table 4.4

All schools should have induction and mentoring programs that meet national and state government guidelines. footnote: Recent new national guidelines were published in 2016 with raised expectations for high quality induction programs. FIND REF!!!

Many teachers are crying out for more support on how to keep the class engaged. They want more opportunities to collaborate with their colleagues on difficult situations in class. More collaboration is teachers own top suggestion on how to improve their skills in this area.  footnote:TALIS 2013 ? Table 4.4

Australia is well below the OECD average in terms of the proportion of teachers giving and receiving feedback generally. footnote:OECD (2014)  - Tables 5.5 and 6.15.
And more than 40 per cent of Australian teachers say they have never observed or given feedback to their colleagues. %Footnote:Freeman, et al. (2014) ? Table 6.4. 
Simonsen et al (2008) recommend a number of self-evaluation checklists teachers can use to assess their own performance against evidence-based criteria. footnote: Insert reference to Simonsen et al (2008), page XX FIND REF!!!

Sadly, student mental health problems are a big contributor to issues in behaviour and learning in Australian classrooms. One in three young Australians experience moderate to high levels of psychological distress including depression and anxiety. footnote: Waters (2011)

There are scores of techniques purported to enhance student engagement and behaviour, but only some of them are proven to work. footnote:For example, see O?Neill and Stephenson (2014). In previous work, the authors identified 55 behaviour management strategies, all of which were being used by Australian primary school teachers. However, only about one-third (18 practices) were backed by evidence that they work. [This should go elsewhere if not related to ITE ? need to check original study again] CHECK REF!!!
Unfortunately, the extent to which ITE courses focus on the evidence-based practices varies greatly.  %FOOTNOTE:This also appears to be true in other countries ? see, for example, Greenberg, Puttman, & Walsh (2014) 

A major 2014 Australian study found that of 19 classroom and behaviour management models taught in undergraduate primary ITE programs, only two came even close to including comprehensive coverage of the evidence-based practices available. %Footnote:O?Neill and Stephenson (2014). The two models with the most evidence-based techniques are ABA (Applied Behavioural Analysis) and PBIS (Positive Behaviour Interventions and Supports), which is derived from ABA.  

A 2010 review of Queensland ITE courses found that many universities allocated only a few hours to behaviour management in the entire course.  %Footnote:Caldwell, B., & Sutton, D. (2010). Review of teacher education and school induction, second report, p. 12

If new teachers do not have a good theoretical framework to draw on in the heat of the moment in class, there may be little guiding their actions. Indeed, some academics warn that \emph{?many teachers may be theoretically blind when it comes to classroom management?}.  %Footnote:Riley et al (2010). A study of the drivers of poor classroom management practices, such as teacher aggression.

The lack of evidence-based techniques for classroom climates reflects broader issues with the lack of evidence based practices taught in ITE generally. This was highlighted by a 2014 review by the Teacher Education Ministerial Advisory Group (TEMAG). %Footnote:See, for example, the Action Now report, TEMAG (2014) 

A major review of initial teacher education by the 2014 TEMAG made a number of recommendations which will help strengthen ITE generally, including the use of evidence based techniques. The changes include more of a focus on achieving successful graduate outcomes, as part of the introduction of a new national accreditation processes for ITE courses.  %Footnote:The Australian Institute for Teaching and School Leadership (AITSL) is now managing a new accreditation process. Providers must submit evidence of program effectiveness, including evidence of the impact on student learning of each program?s graduates. Graduate teachers must also build a portfolio to demonstrate their ?classroom readiness?. The reforms aim to drive improvements by putting more pressure on providers to deliver high quality outcomes, i.e. proficient teaching graduates who have a strong positive impact on their students? learning. Action Now report, TEMAG (2014)
Trainee teachers need a lot more than a theoretical understanding of classroom environments if they are to be properly prepared for stressful classroom situations. They must ?be able to demonstrate the necessary skills, not just know and understand the topic?.  footnote:Caldwell and Sutton (2010), p. 11.

A Queensland review of initial teacher education in 2010 highlighted the need for better in-school experience to equip teachers with behaviour management skills. footnote:Caldwell and Sutton (2010)
%Footnote:Headden 2014 check this reference in Queensland Report of the Teacher Education Implementation Taskforce, 2012 CHECK REF!!!
There are numerous textbooks and guides on how to ?manage? a classroom and student behaviour, but not all are based on sound evidence. Footnote:O?Neill et al (2014) ADD in ; FIND REF!!!
Teachers do not have time to digest a 1200-page handbook on classroom management, let alone explore the evidence base behind each technique.

Data is limited on the extent to which government support is well targeted, however there are some indications it could be improved. A 2013 West Australian Auditor-General?s report on student behaviour found that ?training for teachers in classroom management ? is not targeted to the schools and teachers with greatest need.?  Footnote:https://audit.wa.gov.au/wp-content/uploads/2014/03/insert2014_04-BehMgt.pdf FIND REF!!!

More nuanced indicators on student engagement should be developed so that better information can be collected. This issue was recognised in a 2014 AITSL report.footnote:2014 AITSL report on student engagement FIND REF!!!
\source{Sullivan et al. (2014)}

\source{Sullivan et al. (2014)}

Australia is on par with other TALIS countries on the proportion of teachers who report losing quite a bit of time because of students interrupting the lesson (32 per cent compared to the TALIS average of 31 per cent). footnote:(TALIS 2013, Table 6.6)

\paragraph{But we know suspensions are damaging:} suspended students are 35 per cent less likely to finish high school and are at increased risk of contact with the justice system. footnote:Granite, 2012; Fabelo, 2011 CHECK REF!!!

\paragraph{We have failed to find effective alternatives:} In-school suspensions, suspension centres and behaviour schools are possible alternatives but they continue to exclude difficult students and have not been fully evaluated. footnote: Moore, 2014; Granite, 2012 CHECK REF!!!
Citations to appear in-line with the text are inserted like: \textcite{Daley-etal-2016-SAPTO}.

\footcite{Daley-etal-2016-Assessing-2016-super-tax-reforms}

To refer to a page number: \textcite[][30]{Daley-etal-2016-SAPTO}.

\footcite[][Chapter~4]{Daley-etal-2016-SAPTO}

Use the plural forms of these commands to cite multiple works (\textcites{Piketty2013}{Leigh-2013-BattlersBillionaires}) at the same point.
