\documentclass{article}
\usepackage{lipsum}
\usepackage[utf8]{inputenc}
\usepackage[T1]{fontenc}
\usepackage{lmodern}
\usepackage[left=7cm, right=7cm]{geometry}
\raggedright

\begin{document}
Better targeting of superannuation tax breaks should be one of the first superannuation items of business in the new Federal Parliament.
The government proposes the \$2 trillion superannuation system is to encourage savings to supplement or substitute for the Age Pension.
Better targeting of superannuation tax breaks should be one of the first items of business.
The government proposes to legislate that the aim of the \$2 trillion superannuation system is to encourage savings to supplement or substitute for the Age Pension.
Better retrospective targeting of superannuation tax breaks should be one of the first items of business in the new Federal Parliament.
The government proposes to legislate that the aim of the \$2 trillion superannuation system is to encourage savings to supplement or substitute for the Age Pension.
Tax breaks should only be available when they serve this policy aim.
Claims that the Budget changes will affect many earners are wrong.
The changes will affect about 4 per cent of superannuation, nearly all of them high-income earners who are unlikely to access the Age Pension. 
Nor are the proposed changes retrospective and no-one suggests they are retrospective.
Rather, the changes will affect taxes paid on future super earnings, and entitlements to make future contributions to super. 
Better targeting of superannuation tax breaks should be one of the first items of business in the new Federal Parliament.
The government proposes to legislate that the aim of the \$2 trillion superannuation system is to encourage savings to supplement or substitute for the Age Pension.
Tax breaks should only be available when they serve this policy aim.
Claims that the Budget changes will affect many low- and middle-income earners are wrong.
The changes will affect about 4 per cent of superannuation, nearly all of them high-income earners who are unlikely to access the Age Pension. 
Nor are the proposed changes retrospective.
Many reforms affect investments made in the past, and no-one suggests they are retrospective.
Rather, the changes will affect taxes paid on future super earnings, and entitlements to make future contributions to super. 
Better targeting of superannuation tax breaks should be one of the first items of business in the new Federal Parliament.
The government proposes to legislate that the aim of the \$2 trillion superannuation system is to encourage savings to supplement or substitute for the Age Pension.
Better retrospective targeting of superannuation tax breaks should be one of the first items of business in the new Federal Parliament.
The government proposes to legislate that the aim of the \$2 trillion superannuation system is to encourage savings to supplement or substitute for the Age Pension.
Tax breaks should only be available when they serve this policy aim.
Claims that the Budget changes will affect many earners are wrong.

The changes will affect about 4 per cent of superannuation, nearly all of them high-income earners who are unlikely to access the Age Pension. 
Nor are the proposed changes retrospective and no-one suggests they are retrospective.
Rather, the changes will affect taxes paid on future super earnings, and entitlements to make future contributions to super. 
Better targeting of superannuation tax breaks should be one of the first items of business in the new Federal Parliament.
The government proposes to legislate that the aim of the \$2 trillion superannuation system is to encourage savings to supplement or substitute for the Age Pension.
Tax breaks should only be available when they serve this policy aim.
Claims that the Budget changes will affect many low- and middle-income earners are wrong.
The changes will affect about 4 per cent of superannuation, nearly all of them high-income earners who are unlikely to access the Age Pension. 
Nor are the proposed changes retrospective.
The government proposes to legislate that the aim of the \$2 trillion superannuation system is to encourage savings to supplement or substitute for the Age Pension.
Tax breaks should only be available when they serve this policy aim.
Claims that the Budget changes will affect many low- and middle-income earners are wrong.
The changes will affect about 4 per cent of superannuation, nearly all of them high-income earners who are unlikely to access the Age Pension. 
Nor are the proposed changes retrospective.
Many reforms affect investments made in the past, and no-one suggests they are retrospective.
Rather, the changes will affect taxes paid on future super earnings, and entitlements to make future contributions to super. 
The changes will affect about 4 per cent of superannuation, nearly all of them high-income earners who are unlikely to access the Age Pension. 
Nor are the proposed changes retrospective.
Many reforms affect investments made in the past, and no-one suggests they are retrospective.
Rather, the changes will affect taxes paid on future super earnings, and entitlements to make future contributions to super. 
The changes will affect about 4 per cent of superannuation, nearly all of them high-income earners who are unlikely to access the Age Pension. 
Nor are the proposed changes retrospective.
Many reforms affect investments made in the past, and no-one suggests they are retrospective.
Rather, the changes will affect taxes paid on future super earnings, and entitlements to make future contributions to super. 
\end{document}

