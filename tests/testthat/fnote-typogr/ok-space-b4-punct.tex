\documentclass{article}

\begin{document}

Some words.


\begin{smallbox}{Reserves do not solve resource adequacy }{box:reserves-do-not-solve-resource-adequacy}
In some countries, governments have built strategic reserves -- generation capacity that is only to be used in emergencies. For strategic reserves to work, there must be a clear distinction between `reserve' generation and normal generation. If the lines are blurred -- for example if market participants believe governments might use reserves regularly rather than only in emergencies%
\footnote{As per AGL's response to the South Australian Government proposal to build a reserve gas generator, \textcite{MacdonaldSmith2017AGLshredsPlans}.}
-- then market participants may build less generation themselves, undermining the end goal.

With strategic reserves, significant generation capacity is sitting idle most of the time and governments can be tempted to slip it into regular use. Australia's RERT operates as a temporary rather than permanent reserve mechanism to avoid this problem.

If used properly, a temporary or permanent reserve provides a safety net, but does not address the resource adequacy of the normal market. If used improperly, it could result in a `slippery slope' towards more and more generation being procured by the market operator. There are other mechanisms better suited to this outcome. 
\end{smallbox}

\end{document}
