\documentclass[b5paper]{scrartcl}\usepackage[]{graphicx}\usepackage[]{color}
%% maxwidth is the original width if it is less than linewidth
%% otherwise use linewidth (to make sure the graphics do not exceed the margin)
\makeatletter
\def\maxwidth{ %
  \ifdim\Gin@nat@width>\linewidth
    \linewidth
  \else
    \Gin@nat@width
  \fi
}
\makeatother

\definecolor{fgcolor}{rgb}{0.345, 0.345, 0.345}
\newcommand{\hlnum}[1]{\textcolor[rgb]{0.686,0.059,0.569}{#1}}%
\newcommand{\hlstr}[1]{\textcolor[rgb]{0.192,0.494,0.8}{#1}}%
\newcommand{\hlcom}[1]{\textcolor[rgb]{0.678,0.584,0.686}{\textit{#1}}}%
\newcommand{\hlopt}[1]{\textcolor[rgb]{0,0,0}{#1}}%
\newcommand{\hlstd}[1]{\textcolor[rgb]{0.345,0.345,0.345}{#1}}%
\newcommand{\hlkwa}[1]{\textcolor[rgb]{0.161,0.373,0.58}{\textbf{#1}}}%
\newcommand{\hlkwb}[1]{\textcolor[rgb]{0.69,0.353,0.396}{#1}}%
\newcommand{\hlkwc}[1]{\textcolor[rgb]{0.333,0.667,0.333}{#1}}%
\newcommand{\hlkwd}[1]{\textcolor[rgb]{0.737,0.353,0.396}{\textbf{#1}}}%
\let\hlipl\hlkwb

\usepackage{framed}
\makeatletter
\newenvironment{kframe}{%
 \def\at@end@of@kframe{}%
 \ifinner\ifhmode%
  \def\at@end@of@kframe{\end{minipage}}%
  \begin{minipage}{\columnwidth}%
 \fi\fi%
 \def\FrameCommand##1{\hskip\@totalleftmargin \hskip-\fboxsep
 \colorbox{shadecolor}{##1}\hskip-\fboxsep
     % There is no \\@totalrightmargin, so:
     \hskip-\linewidth \hskip-\@totalleftmargin \hskip\columnwidth}%
 \MakeFramed {\advance\hsize-\width
   \@totalleftmargin\z@ \linewidth\hsize
   \@setminipage}}%
 {\par\unskip\endMakeFramed%
 \at@end@of@kframe}
\makeatother

\definecolor{shadecolor}{rgb}{.97, .97, .97}
\definecolor{messagecolor}{rgb}{0, 0, 0}
\definecolor{warningcolor}{rgb}{1, 0, 1}
\definecolor{errorcolor}{rgb}{1, 0, 0}
\newenvironment{knitrout}{}{} % an empty environment to be redefined in TeX

\usepackage{alltt}



\let\oldLaTeX\LaTeX
\renewcommand{\LaTeX}{\textrm{\oldLaTeX}}

\usepackage{booktabs}
\usepackage{float}
\usepackage{geometry}
\usepackage{graphicx}
\usepackage{lmodern}
\usepackage{microtype}
\usepackage[utf8]{inputenc}
\usepackage[T1]{fontenc}
% Font:
\usepackage{tgpagella,eulervm}
\usepackage{tabularx}
\usepackage{listings}
\usepackage{lstautogobble}
\usepackage{xcolor}
\usepackage{longtable}
\definecolor{Orange}{HTML}{F68B33}
\definecolor{DarkOrange}{HTML}{D4582A}
\usepackage{bold-extra}
\lstset{
language=[LaTeX]TeX,
basicstyle=\ttfamily\small,
commentstyle=\color{black!60}\ttfamily,
texcsstyle=*\color{blue},
moretexcs={setlength,chapter,subsection,subsubsection,appendix,addbibresource,FloatBarrier,contentspage,units,includegraphics,noteswithsource,notes,source,bottomrule,midrule,toprule,footcite,footcites,textcite,textcites,Chapref,topref,Vref,Cref,vpageref,acknowledgements,cmidrule},
columns=flexible,
breaklines=true,
backgroundcolor=\color{black!5},
literate={~} {\(\sim\)}{1},
moredelim = [is][\slshape\color{violet}]{@@}{@@},
classoffset=1,
morekeywords={grattan}, keywordstyle = \color{Orange}\bfseries, 
classoffset=0,
belowskip=0.5\baselineskip,
escapechar={\^*}{*}
}

\newcommand*{\defi}[1]{\textbf{#1}}

\title{Extracting arguments from \LaTeX{}}
\IfFileExists{upquote.sty}{\usepackage{upquote}}{}
\begin{document}

\section{Overview}
This document describes the necessary features of a function that extracts arguments of a \LaTeX{} command, 
and its implementation in the R package \lstinline!TeXCheckR!.

\section{}
A simple \LaTeX{} command has the following structure in a document.

\begin{lstlisting}
\@@command@@{@@first argument@@}
\end{lstlisting}

Na\"ively, a command's argument is the text between the first opening brace and the first closing brace following \lstinline!command!. 
A simple implementation would be

\begin{knitrout}\small
\definecolor{shadecolor}{rgb}{0.969, 0.969, 0.969}\color{fgcolor}\begin{kframe}
\begin{alltt}
\hlstd{extractor} \hlkwb{<-} \hlkwa{function}\hlstd{(}\hlkwc{x}\hlstd{,} \hlkwc{command_name}\hlstd{) \{}
  \hlkwd{gsub}\hlstd{(}\hlkwd{paste0}\hlstd{(}\hlstr{"\textbackslash{}\textbackslash{}\textbackslash{}\textbackslash{}"}\hlstd{, command_name,} \hlstr{"\textbackslash{}\textbackslash{}\{(.*?)\textbackslash{}\textbackslash{}\}"}\hlstd{),} \hlstr{"\textbackslash{}\textbackslash{}1"}\hlstd{, x)}
\hlstd{\}}
\end{alltt}
\end{kframe}
\end{knitrout}

This implementation can even be extended to the \(n\)th argument:

However, no regular expressions can substitute for a parser. 
In this case, the extractor will fail to capture the following:

\begin{lstlisting}
\emph{Some \emph{nested} emphasis.}
\end{lstlisting}

The presence of such expressions leads to the next problem: specification. 
What should the idea function return given the above code?
The answer ultimately depends on the purpose of the extractor, but hereupon the specification for the extractor is that the function should return:

\begin{enumerate}
	\item[] For each instance of \lstinline!\@@command@@!, 
	\begin{enumerate}
		\item the contents of that command, including any nested instances of the command
		\item (other information)
	\end{enumerate}
\end{enumerate}

Thus, the above extract should return both 
\begin{quote}\lstinline!nested!\end{quote} and \begin{quote}\lstinline!Some \emph{nested} emphasis.}!.\end{quote}

\section{Parsing for \TeX{} groups}
A \TeX{} group is any enclosure of braces. 
The \defi{level} of a \TeX{} group is 0 at text not enclosed by braces and \(n+1\) for any enclosure started in a group at level \(n\).
Thus, the following document is underlined by its \TeX{} group level.

\begin{lstlisting}
A{B{CD}E} F{G}
0.1.22.1.00.1.
\end{lstlisting}
(The \TeX{} group level is not well-defined at braces.)

Clearly, identifying the \TeX{} group level is crucial to overcoming the flaw in the na\"ive method. 
Luckily, given a collection of characters, identifying the \TeX{} group level is simple arithmetic:
\begin{knitrout}\small
\definecolor{shadecolor}{rgb}{0.969, 0.969, 0.969}\color{fgcolor}\begin{kframe}
\begin{alltt}
\hlkwd{cumsum}\hlstd{(chars} \hlopt{==} \hlstr{"\{"}\hlstd{)} \hlopt{-} \hlkwd{cumsum}\hlstd{(chars} \hlopt{==} \hlstr{"\}"}\hlstd{)} \hlopt{+} \hlstd{(chars} \hlopt{==} \hlstr{"\}"}\hlstd{)}
\hlkwd{cumsum}\hlstd{(chars} \hlopt{==} \hlstr{"\{"}\hlstd{)} \hlopt{-} \hlkwd{cumsum}\hlstd{(chars} \hlopt{==} \hlstr{"\}"}\hlstd{)} \hlopt{-} \hlstd{(chars} \hlopt{==} \hlstr{"\{"}\hlstd{)}
\end{alltt}
\end{kframe}
\end{knitrout}
which identifies the groups respectively. 

\begin{lstlisting}
A{B{CD}E} F{G}
01122221100111
00112211000010
\end{lstlisting}

The first parsing is preferable, as it makes it possible to identify zero-length groups:

\begin{lstlisting}[columns=fixed]
               chars :A{B{CD}E} F{G} H{}
cumsum(chars == "{") :011222222223333344
cumsum(chars == "}") :000000112222233334
      (chars == "}") :000000101000010001
               parse :011222211001110011
               chars :A{B{CD}E} F{G} H{}
\end{lstlisting}

Thus, the function \lstinline[language=R]!parse_tex!:


\begin{knitrout}\small
\definecolor{shadecolor}{rgb}{0.969, 0.969, 0.969}\color{fgcolor}\begin{kframe}
\begin{alltt}
\hlstd{parse_tex} \hlkwb{<-} \hlkwa{function}\hlstd{(}\hlkwc{tex_lines}\hlstd{) \{}
  \hlcom{# strsplit returns a list, each element}
  \hlcom{# corresponding to a line}
  \hlstd{chars} \hlkwb{<-} \hlkwd{unlist}\hlstd{(}\hlkwd{strsplit}\hlstd{(tex_lines,} \hlkwc{split} \hlstd{=} \hlstr{""}\hlstd{))}
  \hlkwd{list}\hlstd{(}\hlkwc{char} \hlstd{= chars,}
       \hlkwc{tex_group} \hlstd{=} \hlkwd{cumsum}\hlstd{(chars} \hlopt{==} \hlstr{"\{"}\hlstd{)} \hlopt{-} \hlkwd{cumsum}\hlstd{(chars} \hlopt{==} \hlstr{"\}"}\hlstd{)} \hlopt{+} \hlstd{(chars} \hlopt{==} \hlstr{"\}"}\hlstd{))}
\hlstd{\}}
\end{alltt}
\end{kframe}
\end{knitrout}

This function is still inadequate for identifying text within a specific \TeX{} group.
Further, the nested structure of \texttt{chars} is lost: the characters \lstinline!{CD}! are simultaneously part of level 1 and 2 (and possibly 0).

To identify unique \TeX{} groups, rather than simply the \TeX{} group level, we have to add columns to the table structure.
Although adding columns is expensive, it is justifiable in most cases as the maximum tex group level is generally in single digits.

Ultimately, to uniquely identify groups, the group counter must be aware of itself outside the group.

The algorithm is:

for each \TeX{} group level \(j\),
\begin{enumerate}
\item scroll through each character whose \TeX{} group level is no larger than \(j\) taking the cumulative sum of all opening braces that occur at the \TeX{} group level \(j\)
\item for each character at group level \(j\), uniquely identify the group using the cumulative sum.
\item for each character whose group level is no \emph{smaller} than \(j\), fill any blank entries.
\end{enumerate}

\section{Optional group interference}
So far we have assumed the \TeX{} group can be easily matched with the command selected.
In constructions like
\begin{lstlisting}
Extract the \textbf{bold} part of this sentence.
\end{lstlisting}
matching the command to the brace is trivial, choose the string of characters after the last backslash preceding the brace.\footnote{Since \LaTeX{} gobbles whitespace in this context, `word preceding' here means `word preceding any whitespace preceding'.}
However, this is not the general case.
Firstly, optional arguments may be present. 
Although the above method of scrolling to the next \TeX{} group will work in most cases, it will not work if there are \TeX{} groups within the optional arguments.
Furthermore, there may be commands (including the target command) within those optional arguments.

As a result, each iteration of the algorithm must be done in each optional argument separately.
Once this is done, the optional arguments can be excluded and the algorithm performed at the top level.
As it is likely that optional arguments will contain few, if any, instances of the arguments,
 we consider each optional level of the document and omit the extraction algorithm if the command is not present at any optional level greater.


\begin{lstlisting}[columns=fixed]
      chars :\A[B \A[X]{D}]{E} \A{\A{F}}
OPT_GRP_LVL :001111122211110000000000000
GROUP_LEVEL :000000000011101110001112221
opt_group_id:001111122211110000000000000
    group_id:000000000011101110001112221
      chars :\A[B \A[X]{D}]{E} \A{\A{F}}
\end{lstlisting}










\end{document}
