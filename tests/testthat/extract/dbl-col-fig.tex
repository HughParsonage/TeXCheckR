
In the morning peak, the average CBD-bound trip in Sydney takes 70 per cent longer than it would in the middle of the night, but around 80 per cent longer in Melbourne.

\FloatBarrier
\doublecolumnfigure{%
\caption{Congestion on CBD commuting trips is very similar in Sydney and Melbourne}\label{fig:aggregate-delay-CBD-commutes}
\units{Increase in travel time relative to free flow}
\includegraphics{atlas/CBD-commutes-time-of-day-1.pdf}
\noteswithsource{Average delay is calculated as the ratio of trip duration at each point throughout the day to the minimum trip duration observed for that route over the sample period. Details of routes used here are available in \Chapref{chap:Routes-sampled}}{Grattan analysis of Google Maps.}%
}{
\caption{The variability of CBD commuting trip times is very similar in Sydney and Melbourne}\label{fig:aggregate-variability-CBD-commutes}
\units{Increase in travel time relative to free-flow, morning and afternoon peaks}
\includegraphics{atlas/boxplot-increase_in_travel_time-by-City-Weekday--MonFri-excl-holiday-1.pdf}
\noteswithsource{Only the maximum trip times for each route-day-am/pm combination are included in this chart.
The boxes cover the 25th to 75th percentiles.
The vertical line in each box lies at the median for each city.
The `whiskers' on each side of the boxes extend no further than \(\pm1.5w\) where \(w\) is the box width.
Observations beyond the lines are plotted as dots. (\textcite[][`Box Plot Statistics']{R-grDevices}.)
}{Grattan analysis of Google Maps.}
}

\doublecolumnfigure{DBL-FIG-3}{DBL-FIG-4}
