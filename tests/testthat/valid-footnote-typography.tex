\documentclass{article}

\newcommand{\gls}[1]{#1}

\begin{document}

While income-contingent repayment benefits students, \gls{HELP} imposes costs on the government. 
Since its introduction in 1989, \gls{HECS} and subsequently \gls{HELP} loans have been indexed to the consumer price index (\gls{CPI}\@).
Indexation ensures that \gls{HELP} balances keep their real value in the face of inflation.
Since the government’s borrowing cost tends to be higher than inflation, the government subsidises  interest costs.%
   \footnote{\gls{HELP} lending, tuition funding, and most other higher education programs are special appropriations from consolidated government revenue.
The government therefore does not borrow specifically for \gls{HELP}, but \gls{HELP} requires it to borrow more than it otherwise would if students were required to pay their student charges upfront.} The slower debtors are to repay, the higher the subsidies. 

\end{document}
